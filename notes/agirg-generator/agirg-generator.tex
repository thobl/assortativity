\documentclass[a4paper,11pt]{article}

\usepackage[american]{babel}
\usepackage[ttscale=.85]{libertine}
\usepackage{libertinust1math}
\usepackage[T1]{fontenc}
\usepackage{microtype}
\usepackage{hyperref}
\usepackage{amsmath}
\usepackage{amssymb}
\usepackage{sectsty}
\usepackage{graphicx}
\usepackage{todonotes}
\usepackage{siunitx}
\usepackage{booktabs}
\usepackage{xcolor}
\usepackage{mathtools}
\usepackage{paralist}

\usepackage{fnpct}
\setfnpct{after-punct-space=-0.15em}

\makeatletter
\g@addto@macro\bfseries{\boldmath}
\makeatother


\begin{document}
\allsectionsfont{\sffamily}

\section{Experiments}
\label{sec:experiments}

\paragraph{what do we want to see}

\begin{itemize}
\item joint distribution
  \begin{itemize}
  \item distribution for random edge; also conditioning on the degree
    of the other endpoint
  \item simple heat map for degrees of edges (easy to understand)
  \item heat map of relative change
  \end{itemize}
\item assortativity values
  \begin{itemize}
  \item real-world assortativity values: compare Pearson to Spearman
    and Kendall
  \item assortativity depending on sigma
  \item assortativity (different variants) for fixed different sigma
    depending on graph size
  \end{itemize}
\item sanity checks for the model: average degree for increasing graph
  size and different values of $\sigma$
\end{itemize}

\paragraph{required graphs}

\begin{itemize}
\item selection of real-world graphs: only strong and weak ple, only
  undirected
\item properties of generated graphs:
  \begin{itemize}
  \item average degree 15
  \item temperature 0
  \item dimension 2
  \item $\tau \in \{2.2, 2.6, 2.8, 3.0, 3.5\}$
  \item $n \in \{12500, 25000, 50000, 100000, 200000\}$
  \item $\sigma \in \{0.2, 0.4, 0.6, 0.8, 1.0, 1.2, 1.4, 1.6, 1.8\}$
  \item this makes $5 \cdot 5 \cdot 9 = 225$ combinations
    $\rightarrow$ 5 seeds for testing, 10 for final run
  \end{itemize}
\end{itemize}

\paragraph{required stats}

\begin{itemize}
\item \texttt{stats} -- basic information (graph size etc)
  plus different assortativity measures
\item \texttt{degree\_distribution} -- degree distribution for vertices
  and edges; use the edge-variant with 25 buckets and look at bucket
  ids $\{0, 6, 12, 18, 24\}$ (the latter allowing to restrict )
\item \texttt{joint\_histogram} -- not sure whether we need this
\item \texttt{joint\_degree\_distr} -- not sure whether we need this
\item \texttt{edge\_degrees} -- not sure whether we need this
\end{itemize}



\subsection{Generating ACLs}
\label{sec:generating-acls}

very briefly describe how to estimate the average degree and how to
generate using geometric jumps


\subsection{Generating AGIRGs}
\label{sec:generating-agirgs}


We want to use the GIRG generator as black box to generate AGIRGs.
For this, we have to choose GIRG-weights such that the resulting GIRG
is a supergraph of the desired AGIRG and then do an additional coin
flip for each edge correcting the probability.

For this, let $w_v$ for all $v \in V$ denote the weights of the AGIRG
and let $W = \sum_{v \in V} w_v$ be the total weight.  Assuming
$w_u \ge w_v$, we have the connection probability
%
\begin{equation*}
  p_{uv} = \min \left\{ 1, \left( \frac{w_u^{\min\{1, \tau - \sigma\}}
        \cdot w_v^\sigma}{W\cdot \rVert x_u - x_v\rVert^d}
    \right)^\alpha \right\}. 
\end{equation*}
%
For the GIRG we are going to generate, we use weights $w_v'$ with a
total weight $W' = \sum_{v \in V} w_v'$ and we have connection
probability
%
\begin{equation*}
  p_{uv}' = \min \left\{ 1, \left( \frac{w_u' \cdot w_v'}{W'\cdot
        \rVert x_u - x_v\rVert^d} \right)^\alpha \right\}.
\end{equation*}
%
Our goal is to choose weights $w_v'$ such that $p_{uv}' \ge p_{uv}$
for every pair of vertices $u, v \in V$.

Without loss of generality, we assume that $w_v \ge 1$ for all
$v \in V$.  If this is not the case, we can simply scale them, which
yields a supergraph and then subsample with the correct weights.

As $w_u \ge 1$ it holds that $w_u^{\min\{1, \tau - \sigma\}} \le w_u$.
Thus, to show $p_{uv}' \ge p_{uv}$, it suffices to show that
$p_{uv}' = 1$ or that
%
\begin{equation}
  \label{eq:desired-inequality}
  \frac{w_u' \cdot w_v'}{W'} \ge \frac{w_u \cdot w_v^\sigma}{W}
\end{equation}

\subsubsection{$\sigma < 1$}

The case $\sigma < 1$ is easy.  Let $w_{\min}$ be the minimum weight
(minimum over all $w_v$) and set
$w_v' = w_v \cdot w_{\min}^{\sigma - 1}$.  As $\sigma < 1$, it holds
that $w_{\min}^{\sigma - 1} \ge w_v^{\sigma - 1}$.  Thus, we get
%
\begin{equation*}
  \frac{w_u' \cdot w_v'}{W'} = 
  \frac{w_u \cdot w_v \cdot w_{\min}^{\sigma - 1}}{W} \ge 
  \frac{w_u \cdot w_v \cdot w_{v}^{\sigma - 1}}{W} =
  \frac{w_u \cdot w_v^\sigma}{W}
\end{equation*}

\subsubsection{$\sigma > 1$}

It remains to handle $\sigma > 1$.  The naive way would be to scale
all weights by the same factor, e.g., by $w_{\max}^{\sigma - 1}$.
However, this would result in weights that are substantially larger
than necessary for two reasons.  First, if $w_u' \cdot w_v' \ge W'$ we
have $p_{uv}' = 1$ and thus even larger weights do not have to be
scaled.  Secondly (and probably more importantly), the smaller weights
are scaled more than necessary.

We deal with this as follows.  We define auxiliary weight $w''$
as
%
\begin{equation*}
  w_v'' = w_v \cdot \min\{W^{1 / (\sigma + 1)}, w_v\}^{(\sigma - 1) / 2}.
\end{equation*}
%
With this, let $W'' = \sum_{v \in V} w_v''$.  We claim that
$w_v' = w_v'' \cdot \frac{W''}{W}$ yields weights such that
$p_{uv}' \ge p_{uv}$.

To see this, first note that we get from $w''$ to $w'$ by scaling
every weight with the same factor $\frac{W''}{W}$.  Thus,
$W' = W'' \cdot \frac{W''}{W}$.  With this, we get
%
\begin{equation*}
  \frac{w_u' \cdot w_v'}{W'} =
  \frac{w_u'' \cdot \frac{W''}{W}  \cdot w_v'' \cdot
    \frac{W''}{W}}{W'' \cdot \frac{W''}{W}}
  = \frac{w_u'' \cdot  w_v''}{W}.
\end{equation*}
%
It remains to plug in the definition of $w_v''$, making a case
distinction for the minimum.  If $w_v > W^{1 / (\sigma + 1)}$, we get
%
\begin{equation*}
  \frac{w_u' \cdot w_v'}{W'} =
  \frac{w_u'' \cdot  w_v''}{W} \ge
  \frac{w_v''^2}{W} = 
  \frac{\left(w_v \cdot W^{\frac{\sigma - 1}{2(\sigma +
          1)}}\right)^2}{W} >
  \frac{\left(W^{\frac{1}{\sigma + 1}} \cdot W^{\frac{\sigma - 1}{2(\sigma +
          1)}}\right)^2}{W} = 1.
\end{equation*}
%
Thus, if $w_v > W^{1 / (\sigma + 1)}$ we get $p_{uv}' = 1 \ge p_{uv}$
and it remains to consider the case $w_v \le W^{1 / (\sigma + 1)}$.
For this, we get $w_v'' = w_v \cdot w_v^{(\sigma - 1) / 2}$.
Moreover, as $w_u \ge w_v$, we also have
$w_u'' \ge w_u \cdot w_v^{(\sigma - 1) / 2}$ (note the $w_v$ instead of
$w_u$ in the last factor).  With this, we obtain 
%
\begin{equation*}
  \frac{w_u' \cdot w_v'}{W'} =
  \frac{w_u'' \cdot  w_v''}{W} \ge
  \frac{w_u \cdot w_v^{(\sigma - 1) / 2} \cdot w_v \cdot w_v^{(\sigma
      - 1) / 2}}{W} =
  \frac{w_u \cdot w_v \cdot w_v^{\sigma - 1}}{W} =
  \frac{w_u \cdot w_v^\sigma}{W},
\end{equation*}
%
which shows that $p_{uv}' \ge p_{uv}$ also holds in this case.

\subsubsection{Average Degree}
\label{sec:average-degree}

Note sure how to best scale the weights to get the desired average
degree.  I currently just generate multiple AGIRGs, scaling the
weights in between to roughly get the desired average degree.

\end{document}
